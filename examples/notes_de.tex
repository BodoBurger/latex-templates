\documentclass[10pt,a4paper]{article}
\usepackage[utf8]{inputenc}
\usepackage[english]{babel}
\usepackage{amsmath}
\usepackage{amsfonts}
\usepackage{amssymb}
\usepackage{lmodern}
\usepackage{kpfonts}

%Fußnoten speichern:
\usepackage{savefnmark}


\newcommand{\inlinecode}{\texttt}

\author{Josef Jedermann}
\title{Notizen zu Latex}

\newtheorem{beweis}{Beweis}[section]

\begin{document}

\maketitle

\begin{abstract}
Kurze Notizen zu Latex-Buch...
\end{abstract}

\tableofcontents

\newpage

\section{Aufbau, Installation}

\begin{itemize}
\item
Latex-Distribution: Miktex oder Tex Live? Tex Live für glossaries besser geeignet?
\item
Grafiken mit Post-Script (S. 22)
\end{itemize}

\section{Grundlagen}

\begin{itemize}
\item
Dokumentenklassen: Standardklassen limitiert, erweiterte Klassen wie KOMA-Script benutzen (S. 38)
\end{itemize}

\section{Textformatierung und Strukturierung}

\subsection{Logische Textauszeichnung}
\begin{itemize}
\item
Im Normalfall \textbackslash emph benutzen.
\item
Für weitere Unterscheidungen \textbackslash newcommand benutzen (S. 47).
\end{itemize}

\subsection{Listen}
S. 54 - 60

\subsection{Umgebungen}

\subsubsection{Zitatumgebungen}
\verb|\begin{quote / quotation / verse} und \end{}-Tag| (S.62).
\begin{quote}
Die Katze lag auf der Matratze.
\end{quote}

\subsubsection{Theoreme}
Für durchnummerierte Definitionen, Bespiele, Beweise etc. kann man mit \verb|\newtheorem{beispiel}{Beispiel}[section]| eine Umgebung erstellen.
\begin{beweis}
Da es ein Dackel ist, ist es ein Hund.
\end{beweis}

\subsubsection{Direkte Ausgabe für Quelltext}
\verb+\verb|Hier können Commandos stehen|+.
Statt der Balken $\|$ kann auch fast jedes andere Zeichen benutzt werden.

\bigskip

Für mehrzeiliges kann man auch \begin{verbatim}
\begin{verbatim} Hier steht
nicht formatierter Text \end{}
\end{verbatim} benutzen.


\subsubsection{Eigene Umgebungen}

Mit \verb|\newenvironment{•}{•}{•}| kann eine eigene Umgebung festgelegt werden (S. 67).


\subsection{Fußnoten}

Fußnoten werden mit \verb+\footnote{Ich bin eine Fußnote}+ erzeugt und sehen so\footnote{Ich bin eine Fußnote}\saveFN{\Name1} aus. Es gibt auch die Möglichkeit, dieselbe Fußnote mehrmals zu verwenden (\verb+\saveFN{\NameFootnote}+), so\useFN{\Name1} nämlich.



\subsection{Querverweise}

\end{document}