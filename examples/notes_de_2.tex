\documentclass[12pt,a4paper]{article}
\usepackage[utf8]{inputenc}
\usepackage[ngerman]{babel}

\usepackage{dsfont} % double stroke letters
%\usepackage{verbatim}
\usepackage{amsmath}
\usepackage{amsfonts}
\usepackage{amssymb}
\usepackage{lmodern}
\usepackage{kpfonts}

% start section within part at 1
% TODO: conflict with hyperref-package
\usepackage{chngcntr}
\counterwithin*{section}{part}

% turn of identation of paragraphs:
\usepackage{parskip}

\usepackage{graphicx}

\usepackage{bm}

\usepackage[autostyle=true]{csquotes}
%\usepackage{copyrightbox} % for adding copyright information to graphs etc.
\usepackage{algorithm,algorithmic}

%\usepackage{hyperref} % for \url{}-comand; include as last package

%\definecolor{LMUgreen}{RGB}{0, 121, 63}
%\definecolor{darkgreen}{RGB}{0, 20, 10}

% new math commands
\newcommand{\of}[1]{\left(#1\right)}
\newcommand{\off}[1]{\left[#1\right]}
\newcommand{\offf}[1]{\left\{#1\right\}}
\newcommand{\oo}[1]{\frac{1}{#1}} % fraction 1/x
\newcommand{\sumin}[1]{\sum_{i=#1}^{n}} % sum from i=#1 to n
\newcommand{\X}{\bm{X}} % bold X for vector RV
\newcommand{\xB}{\bm{x}} % bold x
\newcommand{\y}{\bm{y}} % bold y
\newcommand{\z}{\bm{z}} % bold z
\newcommand{\aB}{\bm{a}} % bold a
\newcommand{\bB}{\bm{b}} % bold b
\newcommand{\h}{\bm{h}} % bold h
\newcommand{\thetaB}{\bm{\theta}} % bold theta
\newcommand{\E}{\mathbb{E}} % expectation
\newcommand{\N}{\mathcal{N}} % Gaussian distribution N
\newcommand{\R}{\mathbb{R}} % R eucledian space
\newcommand{\iid}[1]{\overset{i.i.d.}{#1}} % i.i.d. over some sign
\newcommand{\mat}[1]{ %short pmatrix command
  \begin{pmatrix}
    #1
  \end{pmatrix}
}

\newcommand{\inlinecode}{\texttt}


%Fußnoten speichern:
%\usepackage{savefnmark}

\author{Walter Frosch}
\title{Notizen zu Papula}

\newtheorem{beweis}{Beweis}[section]

\begin{document}

%\maketitle

% \begin{abstract}
% Kurze Notizen zu Papula (2014) Mathematik für Ingenieure und Naturwissenschaften Band 1.
% \end{abstract}

%\tableofcontents

\newpage

\part{Allgemeine Grundlagen}

\section{Mengenlehre}

\section{Menge der reellen Zahlen}

\section{Gleichungen}
\begin{itemize}
  \item biquadratische Gleichung
  \item Betragsgleichungen: Fallunterscheidungen zur Lösung
\end{itemize}

\section{Ungleichungen}

\section{Lineare Gleichungssysteme}

\begin{itemize}
  \item Gauß-Algorithmus
\end{itemize}

\section{Binomische Lehrsatz}

\begin{align*}
  (a + b)^n = \sum_{k=0}^n \binom{n}{k} a^{n-k} b^k
\end{align*}

\begin{itemize}
  \item Pascalsches Dreieck
\end{itemize}

%%%%%%%%%%%%%%%
\part{Vektoralgebra}

\section{Grundbegriffe}


\section{Vektorrechnung in der Ebene}
Skalarprodukt: \( \bm{a} \cdot \bm{b} = ab \cdot \cos \phi \)


\section{Vektorrechnung im 3-dimensionalen Raum}
Neue Begriffe: \emph{Vektorprodukt} (zwei Vektoren) und \emph{Spatprodukt} (drei Vektoren).

Orthogonale Vektoren: \( \aB \cdot \bB = 0 \)

Winkel zwischen zwei Vektoren: \( \cos \phi = \frac{\aB \bB}{|\aB| |\bB|} \)

\textbf{Projektion eines Vektors}: \( \bB_a = \left(\frac{\aB \bB}{|\aB|^2}\right) \aB \)

Der Betrag des Vektorprodukts entspricht dem Flächeninhalt des von den Vektoren aufgespannten Parallelogramms.

\textbf{Spatprodukt}: Skalarprodukt mit einem Vektorprodukt; Betrag entspricht dem Volumen des
aufgespannten Spats; wenn 0, dann sind die drei Vektoren \emph{koplanar}.

\section{Anwendungen in der Geometrie}
\begin{itemize}
  \item Abstand Punkt von einer Geraden
  \item Abstand zweier windschiefer Geraden
  \item Schnittpunkt/-winkel zweier Geraden
\end{itemize}

\part{Funktionen und Kurven}

\section{Definition und Darstellung einer Funktion}
\begin{itemize}
  \item[x:] Argument
  \item[y:] Funktionswert
  \item[D:] Definitionsbereich
  \item[W:] Wertebereich oder Wertevorrat
\end{itemize}

\section{Allgemeine Funktionseigenschaften}
\begin{itemize}
  \item Nullstellen
  \item Symmetrie
  \item Monotonie
  \item Periodizität
  \item Inverse
\end{itemize}

\section{Koordinatentransformation}


\section{Grundlagen}

\begin{itemize}
\item
Dokumentenklassen: Standardklassen limitiert, erweiterte Klassen wie KOMA-Script
benutzen (S. 38)
\end{itemize}

\section{Textformatierung und Strukturierung}

\subsection{Logische Textauszeichnung}
\begin{itemize}
\item
Im Normalfall \textbackslash emph benutzen.
\item
Für weitere Unterscheidungen \textbackslash newcommand benutzen (S. 47).
\end{itemize}

\subsection{Listen}
S. 54 - 60

\subsection{Umgebungen}

\subsubsection{Zitatumgebungen}
\verb|\begin{quote / quotation / verse} und \end{}-Tag| (S.62).
\begin{quote}
Die Katze lag auf der Matratze.
\end{quote}

\subsubsection{Theoreme}
Für durchnummerierte Definitionen, Bespiele, Beweise etc. kann man mit
\verb|\newtheorem{beispiel}{Beispiel}[section]| eine Umgebung erstellen.
\begin{beweis}
Da es ein Dackel ist, ist es ein Hund.
\end{beweis}

\subsubsection{Direkte Ausgabe für Quelltext}
\verb+\verb|Hier können Commandos stehen|+.
Statt der Balken $\|$ kann auch fast jedes andere Zeichen benutzt werden.

\bigskip

Für mehrzeiliges kann man auch \begin{verbatim}
\begin{verbatim} Hier steht
nicht formatierter Text \end{}
\end{verbatim} benutzen.


\subsubsection{Eigene Umgebungen}

Mit \verb|\newenvironment{•}{•}{•}| kann eine eigene Umgebung festgelegt werden (S. 67).


\subsection{Querverweise}

\end{document}
